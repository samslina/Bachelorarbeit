\documentclass[
   ngerman          % neue deutsche Rechtschreibung
  ,a4paper          % Papiergrösse
  ,12pt             % Schriftgrösse
  ,pdftex
%  ,disable         % Todo-Markierungen auschalten
]{scrreprt}

\usepackage[utf8]{inputenc}        % UTF-8 codierte Dateien
                                   % Dieses Dokument ist unter Unix erstellt, daher
                                   % wird diese Input-Codierung benutzt.
\usepackage[T1]{fontenc}

\usepackage{bericht}

\usepackage{gradientframe}

\usepackage{colortbl}

\usepackage{rotating}

\usepackage{eurosym}

\usepackage{paralist}

\usepackage{enumitem}

\usepackage{rotating}

\usepackage{multirow}

\usepackage{multicol} 

\usepackage{arydshln}

\usepackage{color}

\usepackage[onehalfspacing]{setspace}

\usepackage{listings}             % Include the listings-package for Code
%%\lstset{language=[Sharp]C} % Set your language (you can change the language for each code-block optionally)

\lstset{
language=csh,
basicstyle=\normalsize\ttfamily,
numbers=left,
numberstyle=\tiny,
numbersep=5pt,
numberstyle=\scriptsize,
tabsize=2,
extendedchars=true,
breaklines=true,
frame=b,
stringstyle=\color{blue}\ttfamily,
showspaces=false,
showtabs=false,
xleftmargin=17pt,
framexleftmargin=17pt,
framexrightmargin=5pt,
framexbottommargin=4pt,
commentstyle=\color{green},
morecomment=[l]{//}, %use comment-line-style!
morecomment=[s]{/*}{*/}, %for multiline comments
showstringspaces=false,
morekeywords={ abstract, event, new, struct,
as, explicit, null, switch,
base, extern, object, this,
bool, false, operator, throw,
break, finally, out, true,
byte, fixed, override, try,
case, float, params, typeof,
catch, for, private, uint,
char, foreach, protected, ulong,
checked, goto, public, unchecked,
class, if, readonly, unsafe,
const, implicit, ref, ushort,
continue, in, return, using,
decimal, int, sbyte, virtual,
default, interface, sealed, volatile,
delegate, internal, short, void,
do, is, sizeof, while,
double, lock, stackalloc,
else, long, static,
enum, namespace, string},
keywordstyle=\color{red},
identifierstyle=\color{black},
backgroundcolor=\color{white},
}

\definecolor{editorGray}{rgb}{0.95, 0.95, 0.95}
\definecolor{editorOcher}{rgb}{1, 0.5, 0} % #FF7F00 -> rgb(239, 169, 0)
\definecolor{editorGreen}{rgb}{0, 0.5, 0} % #007C00 -> rgb(0, 124, 0)

\usepackage{minted}

\usepackage{float}
\restylefloat{figure}


\definecolor{javared}{rgb}{0.6,0,0} % for strings
\definecolor{javagreen}{rgb}{0.25,0.5,0.35} % comments
\definecolor{javapurple}{rgb}{0.5,0,0.35} % keywords
\definecolor{javadocblue}{rgb}{0.25,0.35,0.75} % javadoc

\renewcaptionname{ngerman}{\abstractname}{Kurzzusammenfassung}


%%%%%%%%%%%%%%%%%%%%%%%%%%%%%%%%%%%%%%%%%%%%%%%%%%%%%%%%%%%%%%%%%%%%%%%%%%%%%%%
%% Angaben zur Arbeit
%%%%%%%%%%%%%%%%%%%%%%%%%%%%%%%%%%%%%%%%%%%%%%%%%%%%%%%%%%%%%%%%%%%%%%%%%%%%%%%

\newcommand{\Autor}{Avelina Ott}
\newcommand{\MatrikelNummer}{4386414}
\newcommand{\Kursbezeichnung}{TINF15B4}

\newcommand{\FirmenName}{WebCommerce GmbH}
\newcommand{\FirmenStadt}{Offenburg}
\newcommand{\FirmenLogoDeckblatt}{\includegraphics[width=3cm]{bilder/DHBW.png}}
\newcommand{\BetreuerFirma}{Markus Schill}
\newcommand{\BetreuerDHBW}{Dr. Werner Geiger}



%%%%%%%%%%%%%%%%%%%%%%%%%%%%%%%%%%%%%%%%%%%%%%%%%%%%%%%%%%%%%%%%%%%%%%%%%%%%%%
%%%% Befehle in der Arbeit
%%%%%%%%%%%%%%%%%%%%%%%%%%%%%%%%%%%%%%%%%%%%%%%%%%%%%%%%%%%%%%%%%%%%%%%%%%%%%%
\newcommand{\edith}{EDITH}
\newcommand{\cms}{Content-Management-System}
\newcommand{\cmsEdith}{CMS EDITH}

\newcommand{\exkurs}[2]{
\begin{leftbar}
	\textbf{Exkurs - } \textit{#1}: #2
\end{leftbar}
}


%%%%%%%%%%%%%%%%%%%%%%%%%%%%%%%%%%%%%%%%%%%%%%%%%%%%%%%%%%%%%%%%%%%%%%%%%%%%%%%%%%%%%

\newcommand{\Was}{Bachelorarbeit}
% Wird auf dem Deckblatt in der Erklärung benutzt

%%%%%%%%%%%%%%%%%%%%%%%%%%%%%%%%%%%%%%%%%%%%%%%%%%%%%%%%%%%%%%%%%%%%%%%%%%%%%%%%%%%%%




% Benutzt man das "biblatex"-Paket, dann muß das hier stehen:
% siehe auch die mit BIBLATEX markierten Zeilen in bericht.sty
\bibliography{literatur}

\begin{document}
\pagenumbering{Roman}
%%%%%%%%%%%%%%%%%%%%%%%%%%%%%%%%%%%%%%%%%%%%%%%%%%%%%%%%%%%%%%%%%%%%%%%%%%%%%%%

\newcommand{\Titel}{Web-to-Print: Export von Terminen in eine kundenspezifische DOCX-Datei}
\newcommand{\AbgabeDatum}{27. August 2018}

\newcommand{\Dauer}{12 Wochen}

\newcommand{\Abschluss}{Bachelor of Science}

\newcommand{\Studiengang}{Angewandte Informatik}

\hypersetup{%%
  pdfauthor={\Autor},
  pdftitle={\Titel},
  pdfsubject={\Was}
}





\begin{titlepage}
\begin{center}
\vspace*{-2cm}
%\FirmenLogoDeckblatt
\hfill\includegraphics[width=4cm]{bilder/DHBW}\\[2cm]
{\Huge \Titel}\\[2cm]
{\Huge\scshape \Was}\\[2cm]
{\large für die Prüfung zum}\\[0.5cm]
{\Large \Abschluss}\\[0.5cm]
{\large des Studienganges \Studiengang}\\[0.5cm]
{\large an der}\\[0.5cm]
{\large Dualen Hochschule Baden-Württemberg Karlsruhe}\\[0.5cm]
{\large von}\\[0.5cm]
{\large\bfseries \Autor}\\[1cm]
{\large \AbgabeDatum}
\vfill
\end{center}
\begin{tabular}{l@{\hspace{2cm}}l}
Bearbeitungszeitraum	         & \Dauer 			\\
%Matrikelnummer	                 & \MatrikelNummer	\\
Kurs			         & \Kursbezeichnung		\\
Ausbildungsfirma	         & \FirmenName			\\
			         %& \FirmenStadt			\\
Betreuer der Ausbildungsfirma	 & \BetreuerFirma		\\
Gutachter/in der Studienakademie	 & \BetreuerDHBW		\\
\end{tabular}
\end{titlepage}

%%%%%%%%%%%%%%%%%%%%%%%%%%%%%%%%%%%%%%%%%%%%%%%%%%%%%%%%%%%%%%%%%%%%%%%%%%%%%%%

% Nur für Bachelorarbeiten einfügen:

\selectlanguage{ngerman}
%\begin{abstract}
%\end{abstract}


\newpage
\addchap{Eidesstattliche Erklärung}
Gemäß § 5(3) der „Studien- und Prüfungsordnung DHBW Technik“ vom 29. 9. 2015 versichere ich hiermit, dass ich meine Studienarbeit mit dem Thema: „\Titel“ selbstständig verfasst und keine anderen als die angegebenen Quellen und Hilfsmittel benutzt habe. Ich versichere zudem, dass die eingereichte elektronische Fassung mit der gedruckten Fassung übereinstimmt.


\vspace{3cm}
\noindent
\underline{\hspace{10cm}}\\\\
Karlsruhe, 27.08.2018\hspace{4cm}\\\\\\

\tableofcontents           % Inhaltsverzeichnis hier ausgeben
\listoffigures             % Liste der Abbildungen
\listoftables              % Liste der Tabellen
\lstlistoflistings         % Liste der Listings


\lstdefinelanguage{Velocity}
{
  %basicstyle=\ttfamily,
  morestring=[s]{"}{"},
  morekeywords={set, foreach, end, if, else},
  otherkeywords={\#, ., get},
  %moredelim=[s][\color{black}]{>}{<},
  moredelim=[s][\color{red}]{\$}{\ },
  %morecomment=[s][\color{red}]{.}{.},
  %moredelim=[s][\color{red}]{\$}{)},
  %moredelim=[s][\color{red}]{\$}{.},
  morecomment=[s][\color{green}]{.}{(},
  commentstyle=\color{green},
  stringstyle=\color{blue},
  keywordstyle=\color{green}
}

\lstdefinelanguage{HTML5}{
        language=html,
        sensitive=true, 
        alsoletter={<>=-},
        otherkeywords={
        % HTML tags
        <html>, <head>, <title>, </title>, <meta, />, </head>, <body>,
        <canvas, \/canvas>, <script>, </script>, <td>, </td>, <tr>, </tr>, </body>, </html>, <!, html>, <style>, </style>, ><
        },  
        ndkeywords={
        % General
        =,
        % HTML attributes
        charset=, id=, width=, height=,
        % CSS properties
        border:, transform:, -moz-transform:, transition-duration:, transition-property:, transition-timing-function:
        },  
        morecomment=[s]{<!--}{-->},
        tag=[s]
}





%\lstset{language=Java,
%	%basicstyle=\fontsize{10}{13}\ttfamily,
%	keywordstyle=\color{javapurple}\bfseries,
%	stringstyle=\color{javared},
%	commentstyle=\color{javagreen},
%	morecomment=[s][\color{javadocblue}]{/**}{*/},
%	numbers=left,
%	numberstyle=\tiny\color{black},
%	stepnumber=1,
%	numbersep=5pt,
%	tabsize=2,
%	showspaces=false,
%	showstringspaces=false,
%	breaklines=true,
%	breakautoindent=true,
%	morekeywords={set, foreach, end}
%}


\lstset{
numbers = left,
frame = single,
framexleftmargin=15pt,
breaklines=true
}

% Jetzt kommt der "eigentliche" Text
\include{abk}              % Abkürzungsverzeichnis

\pagenumbering{arabic}

\chapter{Einleitung}\label{einleitung}

\section{Problemstellung} \label{problemstellung}
--> Warum gibt es den Bedarf für das Produkt?


%Mit dem Projekt SESAM unterstützt das Erzbischöfliche Ordinariat Freiburg seine Einrichtungen bei der Realisierung eines Internetauftritts. Zum Einsatz kommt dabei das von Web Commerce entwickelte CMS EDITH, welches von derzeit ca. 500 SESAM-Mandanten genutzt wird. Ein großer Teil der Mandanten verwaltet mit EDITH die Gottesdienst-Termine und gibt diese auf dem eigenen Internetauftritt aus. Neben der Website sollen die in der Datenbank vorhandenen Termindaten auch in den individuell gestalteten Pfarrblättern verwendet werden. Bislang steht dazu eine einfach gehaltene Exportfunktion zur Verfügung, die jedoch nur wenig Gestaltungsmöglichkeiten bietet. Demgegenüber stehen zahlreiche Pfarrbriefvarianten, die sich hinsichtlich Struktur und Formatierung teilweise erheblich unterscheiden. Um den individuellen Ansprüchen der Pfarrblattgestaltung gerecht zu werden, muss das Exportergebnis daher manuell aufwändig nachbearbeitet werden. Dafür werden i.d.R. die Programme Microsoft Word, Microsoft Publisher oder Adobe InDesign eingesetzt.
%Im Rahmen der Bachelorarbeit soll eine neue Exportfunktion konzipiert und in Form eines Prototyps umgesetzt werden. Der Export soll dabei eine möglichst individuelle Ausgabe der Termindaten ermöglichen, sodass deutlich weniger Nachbearbeitungsaufwand für die Mitarbeiterinnen und Mitarbeiter der kirchlichen Einrichtungen anfällt. Im Rahmen der Projektarbeit 3 wurden bereits die Anforderungen an einen solchen Export ermittelt und eine Machbarkeitsanalyse durchgeführt. Dabei stellte sich heraus, dass sich das Dateiformat „DOCX“ in diesem Anwendungskontext sehr gut eignet. 


\section{Motivation} \label{motivation}


\section{Zielsetzung} \label{zielsetzung}
--> Wofür soll das Produkt verwendet werden?
%Erwartet wird der Prototyp einer Komponente für das CMS EDITH, der einen individuell gestaltbaren Export von Terminen für einen Pfarrbrief ermöglicht. In einer Evaluation soll nach der Fertigstellung des Prototyps geprüft werden, ob die an den Export gestellten Anforderungen erfüllt werden können.


\section{Vorgehensweise} \label{vorgehensweise}


\include{heutigerStand}
\chapter{Anforderungsanalyse} \label{anforderungsanalyse}

--> Welche Spezifikation gilt für das Produkt und wie wurde die Spezifikation ermittelt?

--> Beschreibung des Produktes anhand der wichtigsten Funktionen und Leistungen? --> Eventuell auch Gebrauchsnutzen?

\section{Funktionale Anforderungen} 



\section{Nicht funktionale Anforderungen}






\chapter{Planung und Konzept} \label{planung}
--> Welcher Entwicklungsprozess wird gewählt und welche Werkzeuge werden verwendet?


\section{Projektplan} \label{projektplan}



\section{Konzept} \label{vorgehensweise}

\chapter{Implementierung}\label{implem} %Niko

\chapter{Evaluierung} \label{evaluierung}

\subsection{Metriken}\label{metriken}
--> wie wird das Produkt erprobt? 
--> Welches Ergebnis zeigt sich?
--> Anhand welcher Metriken wird evaluiert?

\subsection{Ergebnis} \label{ergebnis}
--> Welche Konsequenzen ergeben sich aus der Erprobung und Evaluierung des Produktes?
\chapter{Zusammenfassung} \label{zusfa}

\section{Fazit}

\section{Ausblick}\label{Ausblick}


%\include{anhang}
\pagenumbering{Roman}

\addcontentsline{toc}{chapter}{Literaturverzeichnis}
\printbibliography



% Ab hier beginnt der Anhang
\appendix
%\addcontentsline{toc}{chapter}{Anhang}
%\appendix
%\appendixtoc
%\newpage

\renewcommand*{\thesection}{\Alph{section}}
\chapter*{\appendixname}
%\include{anhang}
%\begin{enumerate}[label=\bfseries \LARGE \Alph*, ref=\Alph*]
%	\item Bilder \label{app:bilder}
%   \item Umfrage \label{app:umfrage}
%  \item manueller Testplan \label{app:maTestplan}
%\end{enumerate}




% Haben Sie das "biblatex"-Paket nicht installiert, benutzen Sie folgendes:
% Ohne das "biblatex"-Paket (s. bericht.sty) produziert folgendes
% "deutsche" Zitate in Literaturverzeichnissen gemaß der Norm DIN 1505,
% Teil 2 vom Jan. 1984.
% Die Zitatmarken werden alphabetisch nach Verfassern
% sortiert und sind durch abgekürzte Verfasserbuchstaben plus
% Erscheinungsjahr in eckigen Klammern gekennzeichnet.

%\bibliographystyle{alphadin}
%\bibliography{literatur}

%%%%%%%%%%%%%%%%%%%%%%%%%%%%%%%%%%%%%%%
% BIBLATEX
% Benutzt man das "biblatex"-Paket, muß man folgendes schreiben:
%\def\refname{Literaturverzeichnis}


%%%%%%%%%%%%%%%%%%%%%%%%%%%%%%%%%%%%%%%





\end{document}
