\chapter{Einleitung}\label{einleitung}

\section{Problemstellung} \label{problemstellung}
--> Warum gibt es den Bedarf für das Produkt?


%Mit dem Projekt SESAM unterstützt das Erzbischöfliche Ordinariat Freiburg seine Einrichtungen bei der Realisierung eines Internetauftritts. Zum Einsatz kommt dabei das von Web Commerce entwickelte CMS EDITH, welches von derzeit ca. 500 SESAM-Mandanten genutzt wird. Ein großer Teil der Mandanten verwaltet mit EDITH die Gottesdienst-Termine und gibt diese auf dem eigenen Internetauftritt aus. Neben der Website sollen die in der Datenbank vorhandenen Termindaten auch in den individuell gestalteten Pfarrblättern verwendet werden. Bislang steht dazu eine einfach gehaltene Exportfunktion zur Verfügung, die jedoch nur wenig Gestaltungsmöglichkeiten bietet. Demgegenüber stehen zahlreiche Pfarrbriefvarianten, die sich hinsichtlich Struktur und Formatierung teilweise erheblich unterscheiden. Um den individuellen Ansprüchen der Pfarrblattgestaltung gerecht zu werden, muss das Exportergebnis daher manuell aufwändig nachbearbeitet werden. Dafür werden i.d.R. die Programme Microsoft Word, Microsoft Publisher oder Adobe InDesign eingesetzt.
%Im Rahmen der Bachelorarbeit soll eine neue Exportfunktion konzipiert und in Form eines Prototyps umgesetzt werden. Der Export soll dabei eine möglichst individuelle Ausgabe der Termindaten ermöglichen, sodass deutlich weniger Nachbearbeitungsaufwand für die Mitarbeiterinnen und Mitarbeiter der kirchlichen Einrichtungen anfällt. Im Rahmen der Projektarbeit 3 wurden bereits die Anforderungen an einen solchen Export ermittelt und eine Machbarkeitsanalyse durchgeführt. Dabei stellte sich heraus, dass sich das Dateiformat „DOCX“ in diesem Anwendungskontext sehr gut eignet. 


\section{Motivation} \label{motivation}


\section{Zielsetzung} \label{zielsetzung}
--> Wofür soll das Produkt verwendet werden?
%Erwartet wird der Prototyp einer Komponente für das CMS EDITH, der einen individuell gestaltbaren Export von Terminen für einen Pfarrbrief ermöglicht. In einer Evaluation soll nach der Fertigstellung des Prototyps geprüft werden, ob die an den Export gestellten Anforderungen erfüllt werden können.


\section{Vorgehensweise} \label{vorgehensweise}

