\chapter{Prototyp} \label{prototyp}
Das Ziel dieser Arbeit ist eine erweiterte Export-Funktion anhand eines Prototyp zu konzipieren und umzusetzen. In diesem Kapitel handelt es sich um das Thema der Prototyp. Dabei wird definiert, was ein Prototyp ist, welche Eigenschaft ein Prototyp besitzen soll und welche Arten es existieren. Zusätzlich wird ein standardisierter Ablauf eines Prototyps geschildert. Am Ende werden Vor- und Nachteile eines solchen Vorgehensmodell erläutert und Alternativen vorgestellt.

\section{Definition} \label{prot:Defintion}
\definition{prototype - 1913}{An original or model after which anything is copied; the pattern of anything to be engraved, or otherwise copied, cast, or the like; a primary form; exemplar;archetype}

\definition{prototype - 2004}{1. An original type, form, or instance serving as a basis or standard for later stages. 2. An original, full-scale, and usually working model of a new product or new version of an existing product. 3. An early, typical example}

\section{Eigenschaften} \label{prot:Eigenschaften}



\section{Prototyp-Arten} \label{prot:Prototyparten}


\section{Vorgehen} \label{prot:Prototyp_vorgehen}


\section{Vor- und Nachteile} \label{prot:Prototyp_vorteile}